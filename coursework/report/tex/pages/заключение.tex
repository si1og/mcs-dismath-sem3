\section*{Заключение}
\addcontentsline{toc}{section}{Заключение}

В ходе выполнения лабораторной работы была разработана программа-калькулятор для работы с конечной арифметикой $Z_8$. Реализованы все требуемые функции: действия малой арифметики (сложение, вычитание, умножение, деление с остатком), действия <<большой>> арифметики для многоразрядных чисел, работа с отрицательными числами, а обработка арифметических выражений со скобками.

\subsection*{Достоинства реализации}

\begin{itemize}
    \item Объектно-ориентированная архитектура с чётким разделением ответственности между классами \texttt{Z8}, \texttt{Operations} и \texttt{CLIUI}.
    \item Парсер арифметических выражений с поддержкой скобок и приоритета действий.
\end{itemize}

\subsection*{Недостатки реализации}

\begin{itemize}
    \item Умножение реализовано через многократное сложение, что при больших множителях приводит к значительным затратам времени. Можно оптимизировать через алгоритм умножения <<в столбик>> с переносами.
    \item Деление выполняется методом последовательного вычитания, что неэффективно для больших делимых. Можно улучшить через <<деление в столбик>>.
    \item Отсутствует сохранение истории вычислений между сеансами работы программы.
\end{itemize}

\subsection*{Масштабируемость}

Архитектура программы обеспечивает гибкость для дальнейшего развития.

\begin{itemize}
    \item Добавление новых действий (возведение в степень, НОД, НОК) сводится к реализации методов в классе \texttt{Operations} и обновлению парсера в \texttt{CLIUI}. Базовая логика системы $Z_8$ остаётся неизменной.
    
    \item Логика арифметических действий полностью отделена от пользовательского интерфейса. Это позволяет заменить консольный интерфейс на графический без изменения классов \texttt{Z8} и \texttt{Operations}.
    
    \item Классы \texttt{Z8} и \texttt{Operations} могут использоваться как библиотека в других проектах, требующих работы с конечными арифметиками.
    
    \item Система легко адаптируется под другие варианты задания~--- достаточно изменить правило <<$+1$>> в методе \texttt{initVariant()}.
\end{itemize}

Таким образом, разработанная программа имеет возможности для масштабирования, что делает её пригодной для дальнейшего развития и использования в более сложных задачах.