\section{Результаты работы программы}

\subsection{Общее описание}

В результате выполнения лабораторной работы была разработана консольная программа-калькулятор для работы с конечной арифметикой $Z_8$. Программа полностью реализует все требования задания:

\begin{itemize}
    \item Арифметические операции: сложение, вычитание, умножение, деление с остатком.
    \item Поддержка многоразрядных чисел (до 8 разрядов).
    \item Работа с отрицательными числами.
    \item Парсинг арифметических выражений со скобками.
    \item Обработка особых случаев: переполнение, деление на ноль, неопределённость $0/0$.
    \item Защита от некорректного пользовательского ввода.
\end{itemize}

Интерфейс программы построен на системе меню с интерактивным режимом калькулятора, позволяющим вводить произвольные арифметические выражения.

\subsection{Демонстрация основных функций}

При запуске программы отображается главное меню (рис.~\ref{fig:Screenshot_2025-12-15_at_11.39.46.png}). В заголовке выводится информация о системе: алфавит, ноль, единица и правило <<$+1$>>.

\screenshot{Screenshot_2025-12-15_at_11.39.46.png}{Главное меню программы}

При выборе пункта <<Информация о системе>> выводится подробная информация о конечной арифметике (рис.~\ref{fig:Screenshot_2025-12-15_at_11.40.21.png}): вариант, основание системы, алфавит, аддитивная и мультипликативная единицы, максимальная разрядность, а также таблица правила <<$+1$>>.

\screenshot{Screenshot_2025-12-15_at_11.40.21.png}{Информация о системе $Z_8$}

При выборе пункта <<Режим калькулятора>> открывается интерактивный режим (рис.~\ref{fig:Screenshot_2025-12-15_at_11.40.33.png}). Программа выводит подсказку о поддерживаемых операциях и примеры выражений.

\screenshot{Screenshot_2025-12-15_at_11.40.33.png}{Режим калькулятора}

\subsection{Демонстрация арифметических операций}

На рисунке~\ref{fig:Screenshot_2025-12-15_at_11.41.05.png} показаны базовые операции малой арифметики:
\begin{itemize}
    \item $a + a = a$ (свойство аддитивной единицы);
    \item $a + b = b$ (свойство аддитивной единицы);
    \item $a \cdot b = a$ (свойство поглощения);
    \item $b \cdot b = b$ (свойство мультипликативной единицы).
\end{itemize}

\screenshot{Screenshot_2025-12-15_at_11.41.05.png}{Базовые арифметические операции}

На рисунке~\ref{fig:Screenshot_2025-12-15_at_11.43.38.png} показаны операции умножения многоразрядных чисел:
\begin{itemize}
    \item $cc \cdot d = gce$ (умножение двузначного числа на цифру);
    \item $gg \cdot b = gg$ (умножение на <<b>>);
    \item $gg \cdot g = hh$ (умножение на <<g>>);
    \item $bca \cdot gg = hbfa$ (умножение трёхзначных чисел).
\end{itemize}

\screenshot{Screenshot_2025-12-15_at_11.43.38.png}{Операции умножения многоразрядных чисел}

На рисунке~\ref{fig:Screenshot_2025-12-15_at_11.44.58.png} показаны операции деления с остатком:
\begin{itemize}
    \item $ba \div g = h$ (деление без остатка);
    \item $bb \div g = h$, остаток $b$ (деление с остатком);
    \item $-bb \div g = -e$, остаток $b$ (деление отрицательного числа);
    \item $-dead \div cc = -df$, остаток $dc$ (деление больших чисел).
\end{itemize}

\screenshot{Screenshot_2025-12-15_at_11.44.58.png}{Операции деления с остатком}

На рисунке~\ref{fig:Screenshot_2025-12-15_at_11.45.58.png} показана работа со скобками:
\begin{itemize}
    \item $(g + b) \cdot g = f$ (сначала сложение, потом умножение);
    \item $(h + e) \div g = h$, остаток $b$ (сложение в скобках, затем деление);
    \item $-(-b) = b$ (двойное отрицание).
\end{itemize}

\screenshot{Screenshot_2025-12-15_at_11.45.58.png}{Операции со скобками}

\subsection{Обработка ошибок и особых случаев}

Программа обеспечивает корректную обработку всех особых случаев (рис.~\ref{fig:Screenshot_2025-12-15_at_22.56.08.png}):

\begin{itemize}
    \item \textbf{Некорректный ввод}: при вводе символа, не принадлежащего алфавиту (например, \texttt{q+a}), выводится сообщение <<Ожидалось число>>.
    \item \textbf{Переполнение}: при превышении максимальной разрядности (например, $-cccccccc - e$) выводится сообщение <<ПЕРЕПОЛНЕНИЕ>>.
    \item \textbf{Неопределённость $0/0$}: при делении нуля на ноль выводится диапазон возможных значений: <<любое число в $[-cccccccc;cccccccc]$>>.
    \item \textbf{Деление на ноль}: при делении ненулевого числа на ноль (например, $b/a$) выводится сообщение <<Пустое множество>>.
\end{itemize}

\screenshot{Screenshot_2025-12-15_at_22.56.08.png}{Обработка ошибок и особых случаев}
