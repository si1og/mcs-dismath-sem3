\section{Математическое описание}

\subsection{<<Малая>> конечная арифметика}

Рассмотрим малую конечную арифметику $\langle Z_8; +, \cdot \rangle$ с алфавитом из восьми символов:
$$Z_8 = \{a, b, c, d, e, f, g, h\}$$

Аддитивной единицей (нулём) является символ <<$a$>>, мультипликативной единицей~--- символ <<$b$>>.

Правило <<$+1$>> задаёт циклический порядок элементов алфавита и определяет операцию прибавления единицы к любому элементу:

\begin{table}[h]
\centering
\begin{tabular}{|c|c|c|c|c|c|c|c|c|}
\hline
$x$ & $a$ & $b$ & $c$ & $d$ & $e$ & $f$ & $g$ & $h$ \\
\hline
$x+1$ & $b$ & $g$ & $a$ & $h$ & $f$ & $c$ & $d$ & $e$ \\
\hline
\end{tabular}
\caption{Правило <<+1>>}
\end{table}

Из данного правила можно задать отношение порядка элементов:

\begin{figure}[h]
\centering
\begin{tikzpicture}[node distance=1.5cm, >=stealth]
    \node (a) at (0,0) {$a$};
    \node (b) at (1.5,0) {$b$};
    \node (g) at (3,0) {$g$};
    \node (d) at (4.5,0) {$d$};
    \node (h) at (6,0) {$h$};
    \node (e) at (7.5,0) {$e$};
    \node (f) at (9,0) {$f$};
    \node (c) at (10.5,0) {$c$};
    
    \draw[->] (a) -- (b);
    \draw[->] (b) -- (g);
    \draw[->] (g) -- (d);
    \draw[->] (d) -- (h);
    \draw[->] (h) -- (e);
    \draw[->] (e) -- (f);
    \draw[->] (f) -- (c);
\end{tikzpicture}
\caption{Отношение порядка в $Z_8$}
\end{figure}

\subsubsection*{Для данной арифметики справедливы следующие свойства:}

\begin{itemize}
    \item элемент <<$a$>> является аддитивной единицей: $\forall x \in Z_8 \; x + a = x$;
    \item свойство поглощения: $\forall x \in Z_8 \; x \cdot a = a$;
    \item элемент <<$b$>> является мультипликативной единицей: $\forall x \in Z_8 \; x \cdot b = x$;
    \item коммутативность: 

    $$\forall a, b \in Z_8 \quad a \cdot b = b \cdot a,$$
    $$\forall a, b \in Z_8 \quad a + b = b + a;$$


    \item ассоциативность:
    $$\forall a, b, c \in Z_8 \quad (a \cdot b) \cdot c = a \cdot (b \cdot c),$$

    \item дистрибутивность:
    $$\forall a, b, c \in Z_8 \quad a \cdot (b + c) = a \cdot b + a \cdot c.$$
    $$\forall a, b, c \in Z_8 \quad (a + b) \cdot c = a \cdot c + b \cdot c.$$
\end{itemize}

Для данной арифметики определены операции сложения и умножения, представленные в виде таблиц ниже.

\subsubsection*{Сложение}

Операция сложения двух элементов $x + y$ определяется как последовательное применение правила <<$+1$>> к элементу $x$ ровно $y$ раз. 

\textbf{Пример 1.} Вычисление $b + a$. Так как а~--- аддитивная единица, то $b + a = b$. 

\textbf{Пример 2.} Для вычисления $g + d$ необходимо трижды применить правило <<$+1$>> к элементу $g$, так как расстояние между $a$ и $d$ в отношении порядка равно трём, т.е., чтобы получить $d$, нужно к $a$ три раза применить правило <<$+1$>>:
$$g \xrightarrow{+1} d \xrightarrow{+1} h \xrightarrow{+1} e$$

Таким образом, $g + d = e$.

В <<малой>> арифметике результат сложения всегда остаётся в пределах алфавита за счёт цикличности:

\begin{table}[h]
\centering
\begin{tabular}{|c|c|c|c|c|c|c|c|c|}
\hline
$+$ & $a$ & $b$ & $g$ & $d$ & $h$ & $e$ & $f$ & $c$ \\
\hline
$a$ & $a$ & $b$ & $g$ & $d$ & $h$ & $e$ & $f$ & $c$ \\
\hline
$b$ & $b$ & $g$ & $d$ & $h$ & $e$ & $f$ & $c$ & $a$ \\
\hline
$g$ & $g$ & $d$ & $h$ & $e$ & $f$ & $c$ & $a$ & $b$ \\
\hline
$d$ & $d$ & $h$ & $e$ & $f$ & $c$ & $a$ & $b$ & $g$ \\
\hline
$h$ & $h$ & $e$ & $f$ & $c$ & $a$ & $b$ & $g$ & $d$ \\
\hline
$e$ & $e$ & $f$ & $c$ & $a$ & $b$ & $g$ & $d$ & $h$ \\
\hline
$f$ & $f$ & $c$ & $a$ & $b$ & $g$ & $d$ & $h$ & $e$ \\
\hline
$c$ & $c$ & $a$ & $b$ & $g$ & $d$ & $h$ & $e$ & $f$ \\
\hline
\end{tabular}
\caption{Таблица сложения в <<малой>> арифметике $Z_8$}
\end{table}

\subsubsection*{Умножение}

Действие умножения $x \cdot y$ определяется как многократное сложение: 
$$x \cdot y = \underbrace{x + x + ... + x}_{y \text{ раз}}$$

Для $y = a$ выполняется $x \cdot a = a$.

Для $y = b$ выполняется $x \cdot b = x$.

В <<малой>> арифметике результат умножения также остаётся в пределах алфавита за счёт цикличности.

\noindent Ниже несколько примеров вычисления элементов таблицы:

\begin{itemize}
    \item Пример 1: $a \cdot g$. Так как для $a$ выполняется свойство поглощения $x \cdot a = a$, то $a \cdot g = a$.
    \item Пример 2: $g \cdot d$. Расстояние между $b$ (мультипликативной единицей) и $d$ в отношении порядка равно двум (это означает, что $g \cdot d = g \cdot (b + b + b)$), следовательно, $g \cdot d = g + g + g = h + g = f$.
    \item Пример 3: $c \cdot f$. Расстояние между $b$ и $f$ в отношении порядка равно пяти, следовательно, $c \cdot f = c + c + c + c + c + c = f + c + c + c + c = e + c + c + c = h + c + c = d + c = g$.
\end{itemize}

\begin{table}[h]
\centering
\begin{tabular}{|c|c|c|c|c|c|c|c|c|}
\hline
$\cdot$ & $a$ & $b$ & $g$ & $d$ & $h$ & $e$ & $f$ & $c$ \\
\hline
$a$ & $a$ & $a$ & $a$ & $a$ & $a$ & $a$ & $a$ & $a$ \\
\hline
$b$ & $a$ & $b$ & $g$ & $d$ & $h$ & $e$ & $f$ & $c$ \\
\hline
$g$ & $a$ & $g$ & $h$ & $f$ & $a$ & $g$ & $h$ & $f$ \\
\hline
$d$ & $a$ & $d$ & $f$ & $b$ & $h$ & $c$ & $g$ & $e$ \\
\hline
$h$ & $a$ & $h$ & $a$ & $h$ & $a$ & $h$ & $a$ & $h$ \\
\hline
$e$ & $a$ & $e$ & $g$ & $c$ & $h$ & $b$ & $f$ & $d$ \\
\hline
$f$ & $a$ & $f$ & $h$ & $g$ & $a$ & $f$ & $h$ & $g$ \\
\hline
$c$ & $a$ & $c$ & $f$ & $e$ & $h$ & $d$ & $g$ & $b$ \\
\hline
\end{tabular}
\caption{Таблица умножения в <<малой>> арифметике $Z_8$}
\end{table}

\subsubsection*{Вычитание}

Операция вычитания $x - y$ определяется как операция, обратная операции сложения: найти такой элемент $z$, что $y + z = x$. Это эквивалентно последовательному применению правила <<$-1$>> (обратного к <<$+1$>>) к элементу $x$ ровно $y$ раз.

Правило <<$-1$>> получается инвертированием правила <<$+1$>>:

\begin{table}[h]
\centering
\begin{tabular}{|c|c|c|c|c|c|c|c|c|}
\hline
$x$ & $a$ & $b$ & $c$ & $d$ & $e$ & $f$ & $g$ & $h$ \\
\hline
$x-1$ & $c$ & $a$ & $f$ & $g$ & $h$ & $e$ & $b$ & $d$ \\
\hline
\end{tabular}
\caption{Правило <<-1>>}
\end{table}

Например, для вычисления $e - d$ необходимо трижды применить правило <<$-1$>> к элементу $e$:
$$e \xrightarrow{-1} h \xrightarrow{-1} d \xrightarrow{-1} g$$

Таким образом, $e - d = g$.

В <<малой>> арифметике результат вычитания всегда остаётся в пределах алфавита за счёт цикличности:

\begin{table}[h]
\centering
\begin{tabular}{|c|c|c|c|c|c|c|c|c|}
\hline
$-$ & $a$ & $b$ & $g$ & $d$ & $h$ & $e$ & $f$ & $c$ \\
\hline
$a$ & $a$ & $c$ & $f$ & $e$ & $h$ & $d$ & $g$ & $b$ \\
\hline
$b$ & $b$ & $a$ & $c$ & $f$ & $e$ & $h$ & $d$ & $g$ \\
\hline
$g$ & $g$ & $b$ & $a$ & $c$ & $f$ & $e$ & $h$ & $d$ \\
\hline
$d$ & $d$ & $g$ & $b$ & $a$ & $c$ & $f$ & $e$ & $h$ \\
\hline
$h$ & $h$ & $d$ & $g$ & $b$ & $a$ & $c$ & $f$ & $e$ \\
\hline
$e$ & $e$ & $h$ & $d$ & $g$ & $b$ & $a$ & $c$ & $f$ \\
\hline
$f$ & $f$ & $e$ & $h$ & $d$ & $g$ & $b$ & $a$ & $c$ \\
\hline
$c$ & $c$ & $f$ & $e$ & $h$ & $d$ & $g$ & $b$ & $a$ \\
\hline
\end{tabular}
\caption{Таблица вычитания в <<малой>> арифметике $Z_8$ (строка минус столбец)}
\end{table}

\subsubsection*{Деление}

Действие деления $x \div y$ определяется как действие, обратное умножению: найти такие элементы $q$ (частное) и $r$ (остаток), что $x = y \cdot q + r$, где $r < y$.

Деление выполняется методом последовательного вычитания: из делимого вычитается делитель до тех пор, пока результат не станет меньше делителя. Количество вычитаний даёт частное, а оставшееся значение~--- остаток.

\paragraph{Особые случаи:}
\begin{itemize}
    \item Деление на ноль ($y = a$) при $x \neq a$ даёт пустое множество: $x \div a = \varnothing$.
    \item Деление нуля на ноль ($a \div a$) даёт неопределённость: результатом является любое число из диапазона $[-cccccccc; cccccccc]$.
    \item $a \div y = a$ для любого $y \neq a$.
    \item $x \div b = x$ для любого $x$.
\end{itemize}

\paragraph{Пример 1:} Вычислим $f \div g$.

Последовательно вычитаем $g$ из $f$:
\begin{itemize}
    \item $f - g = h$, $h \geq g$, продолжаем
    \item $h - g = g$, $g \geq g$, продолжаем
    \item $g - g = a$, $a < g$, останавливаемся
\end{itemize}

Выполнено 3 вычитания, остаток $a$, то есть, выполнилнено деление нацело. Следовательно, $f \div g = d$.

\paragraph{Пример 2:} Вычислим $c \div d$.

Последовательно вычитаем $d$ из $c$:
\begin{itemize}
    \item $c - d = h$, $h \geq d$, продолжаем
    \item $h - d = b$, $b < d$, останавливаемся
\end{itemize}

Выполнено 2 вычитания, остаток $b$. Следовательно, $c \div d = g$, остаток $b$.

\subsection{<<Большая>> конечная арифметика}

<<Большая>> конечная арифметика расширяет малую, позволяя работать с многоразрядными числами. Числа записываются как последовательности символов алфавита, где старшие разряды располагаются слева. Например, в такой арифметике можно записать такие числа как $bgg$ или $caa$.

\subsubsection*{Алгебраическая структура}

Так как необходимо реализовать калькулятор <<большой>> конечной арифметики на базе <<малой>> с поддержкой чисел до 8 разрядов включительно, алгебраическая структура данной арифметики следующая:
$$\langle Z_8^8; +, \cdot \rangle.$$

Множество $Z_8^8$ включает как положительные, так и отрицательные числа. Отрицательные числа записываются с префиксом <<$-$>>, например: $-bg$, $-caa$.

\subsubsection*{Сложение многозначных чисел}

Сложение выполняется поразрядно, справа налево, с учётом переноса. Перенос возникает, когда при сложении двух цифр происходит переход через ноль (элемент <<$a$>>), то есть когда сумма <<переполняет>> алфавит.

\paragraph{Таблица переносов.}

В таблице ниже символ <<$b$>> обозначает наличие переноса в старший разряд, <<$a$>>~--- отсутствие переноса.

\noindent Приведём несколько примеров:

\begin{itemize}
    \item Пример 1: $carry(d, e)$. По таблице сложения $d + e = a$. Так как результат <<меньше>> обоих слагаемых (произошёл переход через $a$), то $carry(d, e) = b$.
    \item Пример 2: $carry(g, d)$. По таблице сложения $g + d = e$. Результат $e$ больше обоих слагаемых, переход через $a$ не произошёл, поэтому $carry(g, d) = a$.
    \item Пример 3: $carry(c, c)$. По таблице сложения $c + c = f$. Результат $f$ меньше слагаемого $c$, произошёл переход через $a$, поэтому $carry(c, c) = b$.
\end{itemize}

\begin{table}[h]
\centering
\begin{tabular}{|c|c|c|c|c|c|c|c|c|}
\hline
$carry$ & $a$ & $b$ & $g$ & $d$ & $h$ & $e$ & $f$ & $c$ \\
\hline
$a$ & $a$ & $a$ & $a$ & $a$ & $a$ & $a$ & $a$ & $a$ \\
\hline
$b$ & $a$ & $a$ & $a$ & $a$ & $a$ & $a$ & $a$ & $b$ \\
\hline
$g$ & $a$ & $a$ & $a$ & $a$ & $a$ & $a$ & $b$ & $b$ \\
\hline
$d$ & $a$ & $a$ & $a$ & $a$ & $a$ & $b$ & $b$ & $b$ \\
\hline
$h$ & $a$ & $a$ & $a$ & $a$ & $b$ & $b$ & $b$ & $b$ \\
\hline
$e$ & $a$ & $a$ & $a$ & $b$ & $b$ & $b$ & $b$ & $b$ \\
\hline
$f$ & $a$ & $a$ & $b$ & $b$ & $b$ & $b$ & $b$ & $b$ \\
\hline
$c$ & $a$ & $b$ & $b$ & $b$ & $b$ & $b$ & $b$ & $b$ \\
\hline
\end{tabular}
\caption{Таблица переносов при сложении}
\end{table}

Перенос возникает тогда и только тогда, когда сумма двух цифр в <<малой>> арифметике <<меньше>> любого из слагаемых (произошёл циклический переход через $a$).

\paragraph{Алгоритм сложения в столбик.}

Пусть даны два числа $A = a_{n-1}a_{n-2}...a_1a_0$ и $B = b_{m-1}b_{m-2}...b_1b_0$. Алгоритм сложения:

\begin{enumerate}
    \item Дополнить меньшее число нулями (символами <<$a$>>) слева до одинаковой длины.
    \item Установить начальный перенос $carry = a$.
    \item Для каждого разряда $i$ от $0$ до $\max(n, m) - 1$:
    \begin{enumerate}
        \item Вычислить сумму $s_i = a_i + b_i + carry$ по таблице сложения <<малой>> арифметики.
        \item Определить новый перенос: если при сложении $a_i + b_i$ или при добавлении $carry$ произошёл переход через <<$a$>>, то $carry = b$, иначе $carry = a$.
    \end{enumerate}
    \item Если после обработки всех разрядов $carry \neq a$, добавить его как старший разряд результата.
\end{enumerate}

\paragraph{Пример.} Вычислим $fc + de$:

\begin{center}
\begin{tabular}{r@{\,}l}
  & $fc$ \\
$+$ & $de$ \\
\hline
\end{tabular}
\end{center}

\begin{itemize}
    \item Разряд 0: $c + e = h$ (по таблице сложения), перенос $= b$ (произошёл переход через $a$)
    \item Разряд 1: $f + d = b$ (по таблице), перенос $= b$; затем $b + b = g$ (добавляем перенос), итого $g$, перенос $= a$
\end{itemize}

Итоговый перенос от разряда 1 равен $b$ (от сложения $f + d$).

Результат: $fc + de = bgh$.

\subsubsection*{Вычитание многозначных чисел}

Вычитание выполняется поразрядно, справа налево, с учётом заимствования. Заимствование возникает, когда цифра уменьшаемого меньше соответствующей цифры вычитаемого.

\paragraph{Алгоритм вычитания в столбик.}

Пусть даны два числа $A = a_{n-1}a_{n-2}...a_1a_0$ и $B = b_{m-1}b_{m-2}...b_1b_0$, причём $A \geq B$. Алгоритм вычитания:

\begin{enumerate}
    \item Дополнить меньшее число нулями (символами <<$a$>>) слева до одинаковой длины.
    \item Установить начальное заимствование $borrow = a$.
    \item Для каждого разряда $i$ от $0$ до $\max(n, m) - 1$:
    \begin{enumerate}
        \item Вычислить разность $d_i = a_i - b_i$ по таблице вычитания <<малой>> арифметики.
        \item Определить заимствование: если $a_i < b_i$, то $newBorrow = b$, иначе $newBorrow = a$.
        \item Вычесть старое заимствование: $d_i = d_i - borrow$. Если при этом $d_i$ было равно $a$ и $borrow = b$, то $newBorrow = b$.
        \item Установить $borrow = newBorrow$.
    \end{enumerate}
\end{enumerate}

\subsubsection*{Умножение многозначных чисел}

Умножение многозначных чисел выполняется методом <<в столбик>>: каждая цифра второго множителя умножается на первый множитель, результаты сдвигаются на соответствующее количество разрядов и складываются.

\paragraph{Алгоритм умножения.}

Пусть даны числа $A$ и $B = b_{m-1}b_{m-2}...b_1b_0$. Алгоритм:

\begin{enumerate}
    \item Инициализировать результат $R = a$.
    \item Для каждой цифры $b_i$ числа $B$ (от младшей к старшей):
    \begin{enumerate}
        \item Вычислить частичное произведение $P_i = A \cdot b_i$ (умножение на одну цифру через многократное сложение).
        \item Сдвинуть $P_i$ на $i$ разрядов влево (добавить $i$ цифр <<a>> справа).
        \item Прибавить $P_i$ к результату: $R = R + P_i$.
    \end{enumerate}
    \item Вернуть $R$.
\end{enumerate}

\paragraph{Пример.} Вычислим $bg \cdot gd$:

Число $gd$ состоит из двух цифр: $d$ (младший разряд) и $g$ (старший разряд). Умножаем $bg$ на каждую цифру отдельно:

\begin{itemize}
    \item $bg \cdot d$: умножаем на младшую цифру $d$, получаем $bg + bg + bg = gh + bg = df$
    \item $bg \cdot g$: умножаем на старшую цифру $g$, получаем $bg + bg = gh$, затем сдвигаем на 1 разряд: $gha$
\end{itemize}

Получаем: $df + gha = gcf$.

Результат: $bg \cdot gd = gcf$.

\subsubsection*{Деление многозначных чисел}

Действие деления в <<большой>> конечной арифметике выполняется методом последовательного вычитания, аналогично делению в <<малой>> арифметике.

\paragraph{Алгоритм деления.}

Пусть даны делимое $A$ и делитель $B \neq a$. Алгоритм:

\begin{enumerate}
    \item Инициализировать частное $Q = a$.
    \item Инициализировать остаток $R = A$.
    \item Пока $R \geq B$:
    \begin{enumerate}
        \item Вычесть делитель из остатка: $R = R - B$.
        \item Увеличить частное на единицу: $Q = Q + b$.
    \end{enumerate}
    \item Вернуть частное $Q$ и остаток $R$.
\end{enumerate}

\paragraph{Деление отрицательных чисел.}

При делении отрицательного числа на положительное остаток должен быть неотрицательным. Если $A < a$ и $B > a$, то выполняется корректировка:
$$A = B \cdot Q' + R', \quad a \leq R' < B$$

Для этого, если при делении $|A|$ на $B$ получен ненулевой остаток $r$, то:
\begin{itemize}
    \item Частное увеличивается на единицу: $Q' = Q + b$
    \item Остаток корректируется: $R' = B - r$
\end{itemize}

\paragraph{Пример.} Вычислим $-dead \div hd$:

\begin{enumerate}
    \item Сначала делим $|{-dead}| = dead$ на $hd$ методом последовательного вычитания:
    \begin{itemize}
        \item $dead - hd = dhha$ (1)
        \item $dhha - hd = ddce$ (2)
        \item $ddce - hd = dddg$ (3)
        \item $dddg - hd = dgfc$ (4)
        \item $dgfc - hd = dggh$ (5)
        \item $\dots$ (продолжаем вычитания)
        \item После $fe$ вычитаний получаем остаток $h$, где $h < hd$.
    \end{itemize}
    Частное $= fe$, остаток $= h$.
    
    \item Так как делимое отрицательное и остаток ненулевой, корректируем:
    \begin{itemize}
        \item Частное: $fe + b = ff$
        \item Остаток: $hd - h = dc$
    \end{itemize}
    
    \item Результат: $-dead \div hd = -ff$, остаток $dc$.
\end{enumerate}