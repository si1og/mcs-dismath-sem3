\section{Особенности реализации}

Программа реализована на языке C++ с использованием стандарта C++17. Архитектура построена на трёх основных классах: \texttt{Z8}, \texttt{Operations} и \texttt{CLIUI}.

\subsection{Константные значения}

Максимальная разрядность чисел задаётся константой:

\begin{lstlisting}[language=C++, caption=Константные значения, basicstyle=\footnotesize\ttfamily]
static const int MAX_DIGITS = 8;
\end{lstlisting}

При превышении данного ограничения операция помечается как переполнение.

\subsection{Класс Z8}

Класс \texttt{Z8} представляет базовую структуру конечной арифметики. Хранит алфавит, правила <<$+1$>> и <<$-1$>>, а также аддитивную и мультипликативную единицы.

\subsubsection*{Структура данных}

\begin{lstlisting}[language=C++, caption=Поля класса Z8, basicstyle=\footnotesize\ttfamily]
class Z8 {
protected:
    std::map<char, char> plusOneRule;   
    std::map<char, char> minusOneRule;  
    std::vector<char> alphabet;          

    char additiveIdentity;       
    char multiplicativeIdentity; 
};
\end{lstlisting}

\subsubsection*{Конструктор и инициализация}

Конструктор инициализирует алфавит и вызывает метод \texttt{initVariant()}, который задаёт правило <<$+1$>> согласно варианту 49.

\begin{itemize}
    \item \textbf{Вход:} номер варианта (по умолчанию 49).
    \item \textbf{Выход:} инициализированный объект системы $Z_8$ с заполненными таблицами правил.
\end{itemize}

\begin{lstlisting}[language=C++, caption=Конструктор Z8, basicstyle=\footnotesize\ttfamily]
Z8::Z8() : additiveIdentity('a'), multiplicativeIdentity('b') {
    alphabet = {'a', 'b', 'c', 'd', 'e', 'f', 'g', 'h'};
    initVariant();
}
\end{lstlisting}

Метод \texttt{initVariant} заполняет таблицы правил <<$+1$>> и <<$-1$>>.

\begin{itemize}
    \item \textbf{Вход:} массив правил перехода для варианта 49.
    \item \textbf{Выход:} заполненные ассоциативные массивы \texttt{plusOneRule} и \texttt{minusOneRule}.
\end{itemize}

\begin{lstlisting}[language=C++, caption=Инициализация правила +1, basicstyle=\footnotesize\ttfamily]
void Z8::initVariant() {
    // a+1=b, b+1=g, c+1=a, d+1=h, e+1=f, f+1=c, g+1=d, h+1=e
    char rule[8] = {'b', 'g', 'a', 'h', 'f', 'c', 'd', 'e'};

    plusOneRule.clear();
    minusOneRule.clear();

    for (int i = 0; i < 8; ++i) {
        char from = alphabet[i];
        char to = rule[i];
        plusOneRule[from] = to;
        minusOneRule[to] = from;
    }
}
\end{lstlisting}

Правило <<$-1$>> строится автоматически как обратное к правилу <<$+1$>>.

\subsubsection*{Операции next и prev}

Методы \texttt{next} и \texttt{prev} реализуют действия $+1$ и $-1$ над одиночными символами.

\begin{itemize}
    \item \textbf{Вход:} символ алфавита.
    \item \textbf{Выход:} следующий (предыдущий) символ по правилу.
\end{itemize}

\begin{lstlisting}[language=C++, caption=Операции next и prev, basicstyle=\footnotesize\ttfamily]
char Z8::next(char c) const {
    auto it = plusOneRule.find(c);
    if (it == plusOneRule.end()) {
        throw std::invalid_argument(
            std::string("Символ '") + c + "' не принадлежит алфавиту");
    }
    return it->second;
}

char Z8::prev(char c) const {
    auto it = minusOneRule.find(c);
    if (it == minusOneRule.end()) {
        throw std::invalid_argument(
            std::string("Символ '") + c + "' не принадлежит алфавиту");
    }
    return it->second;
}
\end{lstlisting}

\subsubsection*{Геттеры}

Методы доступа к полям класса.

\begin{itemize}
    \item \texttt{getZero()}~--- \textbf{Вход:} объект $Z_8$. \textbf{Выход:} символ аддитивной единицы (<<$a$>>).
    \item \texttt{getOne()}~--- \textbf{Вход:} объект $Z_8$. \textbf{Выход:} символ мультипликативной единицы (<<$b$>>).
    \item \texttt{getAlphabet()}~--- \textbf{Вход:} объект $Z_8$. \textbf{Выход:} вектор символов алфавита.
    \item \texttt{getBase()}~--- \textbf{Вход:} объект $Z_8$. \textbf{Выход:} основание системы (8).
\end{itemize}

\begin{lstlisting}[language=C++, caption=Геттеры класса Z8, basicstyle=\footnotesize\ttfamily]
char Z8::getZero() const { return additiveIdentity; }
char Z8::getOne() const { return multiplicativeIdentity; }
const std::vector<char>& Z8::getAlphabet() const { return alphabet; }
int Z8::getBase() const { return static_cast<int>(alphabet.size()); }
\end{lstlisting}

\subsubsection*{Проверки и вывод информации}

Дополнительные методы класса \texttt{Z8}:

\begin{itemize}
    \item \texttt{isZero(char c)}~--- проверяет, является ли символ аддитивной единицей.
    \begin{itemize}
        \item \textbf{Вход:} цифра.
        \item \textbf{Выход:} истина, если аддитивная единица, иначе ложь.
    \end{itemize}
    \item \texttt{isOne(char c)}~--- проверяет, является ли символ мультипликативной единицей.
    \begin{itemize}
        \item \textbf{Вход:} цифра.
        \item \textbf{Выход:} истина, если мультипликативная единица, иначе ложь.
    \end{itemize}
    \item \texttt{isValidChar(char c)}~--- проверяет принадлежность символа алфавиту.
    \begin{itemize}
        \item \textbf{Вход:} цифра.
        \item \textbf{Выход:} истина, если мультипликативная единица, иначе ложь.
    \end{itemize}
    \item \texttt{printPlusOneTable()}~--- выводит таблицу правила <<$+1$>> в консоль.
    \begin{itemize}
        \item \textbf{Вход:} алфавит.
        \item \textbf{Выход:} таблица <<$+1$>>.
    \end{itemize}
    \item \texttt{printInfo()}~--- выводит полную информацию о системе: вариант, алфавит, единицы, таблицу правила.
    \begin{itemize}
        \item \textbf{Вход:} алфавит, аддитивная единица, мультипликативная единица, максимальная разрядность.
        \item \textbf{Выход:} информация о системе $Z_8$.
    \end{itemize}
\end{itemize}

\subsection{Класс Operations}

Класс \texttt{Operations} наследуется от \texttt{Z8} и реализует арифметические операции над многоразрядными числами.

\subsubsection*{Структуры результатов}

Для хранения результатов операций используются две структуры:

\begin{lstlisting}[language=C++, caption=Структуры результатов, basicstyle=\footnotesize\ttfamily]
struct OperationResult {
    std::string value;
    bool isOverflow;
    bool isNegative;

    OperationResult() : isOverflow(false), isNegative(false) {}
};

struct DivisionResult {
    std::string quotient;
    std::string remainder;
    bool isOverflow;
    bool isDivByZero;
    bool isZeroByZero;

    DivisionResult() 
        : isOverflow(false), isDivByZero(false), isZeroByZero(false) {}
};
\end{lstlisting}

\subsubsection*{Вспомогательные методы}

Класс \texttt{Operations} содержит ряд вспомогательных методов для работы с числами:

\begin{itemize}
    \item \texttt{abs(num)}~--- возвращает абсолютное значение числа (удаляет знак минус).
    \begin{itemize}
        \item \textbf{Вход:} число.
        \item \textbf{Выход:} модуль числа.
    \end{itemize}
    \item \texttt{isZeroNumber(num)}~--- проверяет, является ли число нулём.
    \begin{itemize}
        \item \textbf{Вход:} число.
        \item \textbf{Выход:} истина, если ноль, иначе ложь.
    \end{itemize}
    \item \texttt{isNegative(num)}~--- проверяет, является ли число отрицательным.
    \begin{itemize}
        \item \textbf{Вход:} число.
        \item \textbf{Выход:} истина, если отрицательное, иначе ложь.
    \end{itemize}
    \item \texttt{negate(num)}~--- меняет знак числа на противоположный.
    \begin{itemize}
        \item \textbf{Вход:} число.
        \item \textbf{Выход:} число с противополеженным знаком.
    \end{itemize}
    \item \texttt{normalize(num)}~--- приводит число к стандартному виду (удаляет ведущие нули, обрабатывает знак).
    \begin{itemize}
        \item \textbf{Вход:} цисло.
        \item \textbf{Выход:} <<нормализованное>> число.
    \end{itemize}
    \item \texttt{isValidNumber(num)}~--- проверяет корректность записи числа.
    \begin{itemize}
        \item \textbf{Вход:} число.
        \item \textbf{Выход:} истина, если запись корректна, иначе ложь.
    \end{itemize}
    \item \texttt{getMaxPositive()}~--- возвращает максимальное положительное число ($cccccccc$).
    \begin{itemize}
        \item \textbf{Вход:} самая большая цифра, максимальная разрядность.
        \item \textbf{Выход:} максимальное положительное число.
    \end{itemize}
    \item \texttt{getMaxNegative()}~--- возвращает минимальное отрицательное число ($-cccccccc$).
    \begin{itemize}
        \item \textbf{Вход:} самая большая цифра, максимальная разрядность.
        \item \textbf{Выход:} минимальное отрицательное число.
    \end{itemize}
\end{itemize}

Метод \texttt{trimLeadingZeros} удаляет ведущие нули из числа.

\begin{itemize}
    \item \textbf{Вход:} строка, представляющая число в системе $Z_8$.
    \item \textbf{Выход:} строка без ведущих нулей (минимум один символ).
\end{itemize}

\begin{lstlisting}[language=C++, caption=Удаление ведущих нулей, basicstyle=\footnotesize\ttfamily]
std::string Operations::trimLeadingZeros(const std::string& num) const {
    if (num.empty()) return std::string(1, getZero());

    size_t start = 0;
    while (start < num.length() - 1 && num[start] == getZero()) {
        ++start;
    }

    return num.substr(start);
}
\end{lstlisting}

Метод \texttt{compareAbs} сравнивает абсолютные значения двух чисел.

\begin{itemize}
    \item \textbf{Вход:} два числа $a$ и $b$.
    \item \textbf{Выход:} $1$ если $|a| > |b|$, $-1$ если $|a| < |b|$, $0$ если равны.
\end{itemize}

\begin{lstlisting}[language=C++, caption=Сравнение по модулю, basicstyle=\footnotesize\ttfamily]
int Operations::compareAbs(const std::string& a, const std::string& b) const {
    std::string normA = trimLeadingZeros(abs(a));
    std::string normB = trimLeadingZeros(abs(b));

    if (normA.length() != normB.length()) {
        return normA.length() > normB.length() ? 1 : -1;
    }

    for (size_t i = 0; i < normA.length(); ++i) {
        if (normA[i] == normB[i]) continue;

        char ca = normA[i];
        char cb = normB[i];

        int posA = 0, posB = 0;
        char cur = getZero();
        for (int j = 0; j < getBase(); ++j) {
            if (cur == ca) posA = j;
            if (cur == cb) posB = j;
            cur = next(cur);
        }

        return posA > posB ? 1 : -1;
    }

    return 0;
}
\end{lstlisting}

\subsubsection*{Инкремент и декремент}

Метод \texttt{increment} прибавляет единицу к многоразрядному числу.

\begin{itemize}
    \item \textbf{Вход:} число.
    \item \textbf{Выход:} число, увеличенное на единицу.
\end{itemize}

\begin{lstlisting}[language=C++, caption=Инкремент числа, basicstyle=\footnotesize\ttfamily]
std::string Operations::increment(const std::string& num) const {
    if (num.empty()) return std::string(1, getOne());

    std::string result = num;

    for (int i = static_cast<int>(result.length()) - 1; i >= 0; --i) {
        char nextChar = next(result[i]);
        result[i] = nextChar;

        if (nextChar != getZero()) {
            return result;
        }
    }

    return std::string(1, getOne()) + result;
}
\end{lstlisting}

Алгоритм проходит по цифрам справа налево, применяя правило $+1$. Если после применения цифра не стала нулём, операция завершена. Иначе продолжаем с переносом.

Метод \texttt{decrement} вычитает единицу из многоразрядного числа. Алгоритм аналогичен инкременту, но использует правило $-1$. При достижении нуля в текущем разряде устанавливается максимальная цифра и заимствование передаётся в старший разряд.

\begin{itemize}
    \item \textbf{Вход:} число.
    \item \textbf{Выход:} число, уменьшенное на единицу.
\end{itemize}

\subsubsection*{Сложение положительных чисел}

Метод \texttt{addPositive} складывает два положительных многоразрядных числа.

\begin{itemize}
    \item \textbf{Вход:} два положительных числа $a$ и $b$.
    \item \textbf{Выход:} сумма $a + b$.
\end{itemize}

\begin{lstlisting}[language=C++, caption=Сложение положительных чисел, basicstyle=\footnotesize\ttfamily]
std::string Operations::addPositive(const std::string& a, 
                                     const std::string& b) const {
    std::string numA = trimLeadingZeros(a);
    std::string numB = trimLeadingZeros(b);

    size_t maxLen = std::max(numA.length(), numB.length());
    while (numA.length() < maxLen) 
        numA = std::string(1, getZero()) + numA;
    while (numB.length() < maxLen) 
        numB = std::string(1, getZero()) + numB;

    std::string result(maxLen, getZero());
    char carry = getZero();

    for (int i = static_cast<int>(maxLen) - 1; i >= 0; --i) {
        char sum = numA[i];
        char newCarry = getZero();

        char counter = numB[i];
        while (counter != getZero()) {
            sum = next(sum);
            if (sum == getZero()) {
                newCarry = next(newCarry);
            }
            counter = prev(counter);
        }

        while (carry != getZero()) {
            sum = next(sum);
            if (sum == getZero()) {
                newCarry = next(newCarry);
            }
            carry = prev(carry);
        }

        result[i] = sum;
        carry = newCarry;
    }

    if (carry != getZero()) {
        result = std::string(1, carry) + result;
    }

    return trimLeadingZeros(result);
}
\end{lstlisting}

Сложение выполняется поразрядно, справа налево. Для прибавления цифры $d$ к сумме применяем операцию \texttt{next()} ровно $d$ раз. При переходе через ноль увеличиваем перенос.

\subsubsection*{Вычитание положительных чисел}

Метод \texttt{subtractPositive} вычитает одно положительное число из другого (при условии $a \geq b$).

\begin{itemize}
    \item \textbf{Вход:} два положительных числа $a \geq b$.
    \item \textbf{Выход:} разность $a - b$.
\end{itemize}

\begin{lstlisting}[language=C++, caption=Вычитание положительных чисел, basicstyle=\footnotesize\ttfamily]
std::string Operations::subtractPositive(const std::string& a, 
                                          const std::string& b) const {
    std::string numA = trimLeadingZeros(a);
    std::string numB = trimLeadingZeros(b);

    if (compareAbs(numA, numB) < 0) {
        throw std::logic_error("subtractPositive: a < b");
    }

    size_t maxLen = numA.length();
    while (numB.length() < maxLen) 
        numB = std::string(1, getZero()) + numB;

    std::string result(maxLen, getZero());
    char borrow = getZero();

    for (int i = static_cast<int>(maxLen) - 1; i >= 0; --i) {
        char diff = numA[i];
        char newBorrow = getZero();

        char counter = numB[i];
        while (counter != getZero()) {
            if (diff == getZero()) {
                newBorrow = next(newBorrow);
                diff = prev(getZero());
            } else {
                diff = prev(diff);
            }
            counter = prev(counter);
        }

        while (borrow != getZero()) {
            if (diff == getZero()) {
                newBorrow = next(newBorrow);
                diff = prev(getZero());
            } else {
                diff = prev(diff);
            }
            borrow = prev(borrow);
        }

        result[i] = diff;
        borrow = newBorrow;
    }

    return trimLeadingZeros(result);
}
\end{lstlisting}

Вычитание аналогично сложению, но используется операция \texttt{prev()}. При достижении нуля берём заимствование из старшего разряда.

\subsubsection*{Умножение на цифру}

Метод \texttt{multiplyByDigit} умножает число на одну цифру через многократное сложение.

\begin{itemize}
    \item \textbf{Вход:} многоразрядное число \texttt{num} и цифра-множитель \texttt{digit}.
    \item \textbf{Выход:} произведение \texttt{num} $\cdot$ \texttt{digit}.
\end{itemize}

\begin{lstlisting}[language=C++, caption=Умножение на цифру, basicstyle=\footnotesize\ttfamily]
std::string Operations::multiplyByDigit(const std::string& num, 
                                         char digit) const {
    if (digit == getZero()) return std::string(1, getZero());
    if (digit == getOne()) return num;

    std::string result(1, getZero());

    char counter = digit;
    while (counter != getZero()) {
        result = addPositive(result, num);
        counter = prev(counter);
    }

    return result;
}
\end{lstlisting}

\subsubsection*{Умножение многозначных чисел}

Метод \texttt{multiplyPositive} реализует умножение <<в столбик>>.

\begin{itemize}
    \item \textbf{Вход:} два положительных многоразрядных числа $a$ и $b$.
    \item \textbf{Выход:} произведение $a \cdot b$.
\end{itemize}

\begin{lstlisting}[language=C++, caption=Умножение многозначных чисел, basicstyle=\footnotesize\ttfamily]
std::string Operations::multiplyPositive(const std::string& a, 
                                          const std::string& b) const {
    if (isZeroNumber(a) || isZeroNumber(b)) {
        return std::string(1, getZero());
    }

    std::string numA = trimLeadingZeros(a);
    std::string numB = trimLeadingZeros(b);

    std::string result(1, getZero());

    for (size_t i = 0; i < numB.length(); ++i) {
        char digit = numB[numB.length() - 1 - i];

        if (digit != getZero()) {
            std::string partial = multiplyByDigit(numA, digit);

            // cдвиг на i разрядов (добавление i нулей)
            for (size_t j = 0; j < i; ++j) {
                partial += getZero();
            }

            result = addPositive(result, partial);
        }
    }

    return trimLeadingZeros(result);
}
\end{lstlisting}

\subsubsection*{Деление с остатком}

Метод \texttt{divide} выполняет целочисленное деление с остатком.

\begin{itemize}
    \item \textbf{Вход:} делимое $a$ и делитель $b$.
    \item \textbf{Выход:} частное $q$ и остаток $r$, такие что $a = b \cdot q + r$, где $0 \leq r < |b|$.
\end{itemize}

\begin{lstlisting}[language=C++, caption=Деление с остатком, basicstyle=\footnotesize\ttfamily]
DivisionResult Operations::divide(const std::string& a, const std::string& b) const {
    DivisionResult res;

    std::string normA = normalize(a);
    std::string normB = normalize(b);

    bool negA = isNegative(normA);
    bool negB = isNegative(normB);
    std::string absA = abs(normA);
    std::string absB = abs(normB);

    if (isZeroNumber(normB)) {
        if (isZeroNumber(normA)) {
            res.isZeroByZero = true;
            res.quotient = "[" + getMaxNegative() + "; " + getMaxPositive() + "]";
            res.remainder = std::string(1, getZero());
        } else {
            res.isDivByZero = true;
            res.quotient = "Пустое множество";
            res.remainder = "Пустое множество";
        }
        return res;
    }

    // 0 / x = 0
    if (isZeroNumber(normA)) {
        res.quotient = std::string(1, getZero());
        res.remainder = std::string(1, getZero());
        return res;
    }

    std::string quotient(1, getZero());
    std::string remainder = absA;

    while (compareAbs(remainder, absB) >= 0) {
        remainder = subtractPositive(remainder, absB);
        quotient = increment(quotient);
    }

    // -a / b с остатком
    if (negA && !negB && !isZeroNumber(remainder) 
        || !negA && negB && !isZeroNumber(remainder)) {
        quotient = increment(quotient);
        remainder = subtractPositive(absB, remainder);
    }

    bool negResult = (negA != negB);

    res.quotient = trimLeadingZeros(quotient);
    res.remainder = trimLeadingZeros(remainder);

    if (negResult && !isZeroNumber(res.quotient)) {
        res.quotient = "-" + res.quotient;
    }

    if (abs(res.quotient).length() > MAX_DIGITS) {
        res.isOverflow = true;
    }

    return res;
}
\end{lstlisting}

Для отрицательного делимого выполняется корректировка: частное увеличивается на единицу, а остаток вычисляется как разность делителя и текущего остатка. Это гарантирует неотрицательность остатка.

\subsection{Класс CLIUI}

Класс \texttt{CLIUI} реализует консольный интерфейс пользователя с поддержкой парсинга арифметических выражений.

\subsubsection*{Структура данных}

Класс хранит умный указатель на объект арифметических операций.

\begin{lstlisting}[language=C++, caption=Поля класса CLIUI, basicstyle=\footnotesize\ttfamily]
class CLIUI {
private:
    std::unique_ptr<Operations> ops;
};
\end{lstlisting}

\subsubsection*{Главный цикл программы}

Метод \texttt{run} реализует бесконечный цикл отображения главного меню.

\begin{itemize}
    \item \textbf{Вход:} запуск программы.
    \item \textbf{Выход:} завершение программы по команде пользователя.
\end{itemize}

\begin{lstlisting}[language=C++, caption=Главный цикл, basicstyle=\footnotesize\ttfamily]
void CLIUI::run() {
    while (true) {
        showMainMenu();
    }
}
\end{lstlisting}

\subsubsection*{Главное меню}

Метод \texttt{showMainMenu} отображает главное меню и обрабатывает выбор пользователя.

\begin{itemize}
    \item \textbf{Вход:} выбор пользователя (символ или цифра).
    \item \textbf{Выход:} переход к выбранному действию или завершение программы.
\end{itemize}

\begin{lstlisting}[language=C++, caption=Главное меню, basicstyle=\footnotesize\ttfamily]
void CLIUI::showMainMenu() {
    clearScreen();

    printHeader("КАЛЬКУЛЯТОР КОНЕЧНОЙ АРИФМЕТИКИ Z8 (Вариант 49)");

    if (!hasSystem()) {
        ops = std::make_unique<Operations>();
    }

    std::cout << "  Алфавит: {a, b, c, d, e, f, g, h}\n";
    std::cout << "  Ноль: " << ops->getZero() 
              << ", Единица: " << ops->getOne() << "\n";
    std::cout << "  Правило +1: a->b->g->d->h->e->f->c->a\n\n";

    printSeparator();
    std::cout << "Главное меню:\n";

    std::cout << "  [1] Информация о системе\n";
    std::cout << "  [C] Режим калькулятора\n";
    std::cout << "  [0] Выход\n";

    std::cout << "\nВаш выбор: ";

    std::string choice;
    std::cin >> choice;

    if (choice == "1") {
        showSystemInfo();
    } else if (choice == "c" || choice == "C") {
        calculatorMode();
    } else if (choice == "0") {
        std::cout << "\n  До свидания!\n";
        exit(0);
    }
}
\end{lstlisting}

\subsubsection*{Парсер выражений}

Парсер реализует разбор арифметических выражений со скобками методом рекурсивного спуска.

\paragraph{Грамматика.}
\begin{itemize}
    \item \texttt{expression} ::= \texttt{term} (('+' | '-') \texttt{term})*
    \item \texttt{term} ::= \texttt{factor} (('*' | '/') \texttt{factor})*
    \item \texttt{factor} ::= \texttt{number} | '(' \texttt{expression} ')' | '-' \texttt{factor}
\end{itemize}

Метод \texttt{parseNumber} считывает число из строки выражения.

\begin{itemize}
    \item \textbf{Вход:} строка выражения и текущая позиция.
    \item \textbf{Выход:} нормализованное число; позиция обновляется.
\end{itemize}

\begin{lstlisting}[language=C++, caption=Парсинг числа, basicstyle=\footnotesize\ttfamily]
std::string CLIUI::parseNumber(const std::string& expr, size_t& pos) {
    skipSpaces(expr, pos);
    
    std::string num;
    
    if (pos < expr.length() && expr[pos] == '-') {
        num += '-';
        ++pos;
    }
    
    while (pos < expr.length() && ops->isValidChar(expr[pos])) {
        num += expr[pos];
        ++pos;
    }
    
    if (num.empty() || num == "-") {
        throw std::runtime_error("Ожидалось число");
    }
    
    return ops->normalize(num);
}
\end{lstlisting}

Метод \texttt{parseFactor} разбирает фактор (число, выражение в скобках или унарный минус).

\begin{itemize}
    \item \textbf{Вход:} строка выражения и текущая позиция.
    \item \textbf{Выход:} вычисленное значение фактора; позиция обновляется.
\end{itemize}

\begin{lstlisting}[language=C++, caption=Вычисление фактора, basicstyle=\footnotesize\ttfamily]
std::string CLIUI::parseFactor(const std::string& expr, size_t& pos) {
    skipSpaces(expr, pos);
    
    bool negate = false;
    if (pos < expr.length() && expr[pos] == '-') {
        size_t nextPos = pos + 1;
        skipSpaces(expr, nextPos);
        if (nextPos < expr.length() && expr[nextPos] == '(') {
            negate = true;
            ++pos;
            skipSpaces(expr, pos);
        }
    }
    
    std::string result;
    
    if (pos < expr.length() && expr[pos] == '(') {
        ++pos;
        result = parseExpression(expr, pos);
        skipSpaces(expr, pos);
        if (pos >= expr.length() || expr[pos] != ')') {
            throw std::runtime_error("Ожидалась закрывающая скобка ')'");
        }
        ++pos;
    } else {
        result = parseNumber(expr, pos);
    }
    
    if (negate) {
        result = ops->negate(result);
    }
    
    return result;
}
\end{lstlisting}

Метод обрабатывает унарный минус перед скобками: $-(expr)$ вычисляется как отрицание результата выражения в скобках.


Метод \texttt{parseTerm} разбирает терм (произведение или частное факторов).

\begin{itemize}
    \item \textbf{Вход:} строка выражения \texttt{expr} и текущая позиция \texttt{pos}.
    \item \textbf{Выход:} вычисленное значение терма; позиция обновляется.
\end{itemize}

\begin{lstlisting}[language=C++, caption=Парсинг терма, basicstyle=\footnotesize\ttfamily]
std::string CLIUI::parseTerm(const std::string& expr, size_t& pos) {
    std::string left = parseFactor(expr, pos);
    
    while (true) {
        skipSpaces(expr, pos);
        
        if (pos >= expr.length()) break;
        
        char op = expr[pos];
        if (op != '*' && op != '/') break;
        
        ++pos;
        std::string right = parseFactor(expr, pos);
        
        if (op == '*') {
            OperationResult res = ops->multiply(left, right);
            left = resultToString(res);
        } else {
            DivisionResult res = ops->divide(left, right);
            if (res.isDivByZero) {
                throw std::runtime_error("Деление на ноль");
            }
            if (res.isZeroByZero) {
                throw std::runtime_error(
                    "Неопределённость 0/0: результат - любое число от " 
                    + ops->getMaxNegative() + " до " + ops->getMaxPositive());
            }
            if (!ops->isZeroNumber(res.remainder)) {
                std::cout << "  (остаток: " << res.remainder << ")\n";
            }
            left = res.quotient;
        }
    }
    
    return left;
}
\end{lstlisting}

Метод \texttt{parseExpression} разбирает выражение (сумму или разность термов).

\begin{itemize}
    \item \textbf{Вход:} строка выражения \texttt{expr} и текущая позиция \texttt{pos}.
    \item \textbf{Выход:} вычисленное значение выражения; позиция обновляется.
\end{itemize}

\begin{lstlisting}[language=C++, caption=Парсинг выражения, basicstyle=\footnotesize\ttfamily]
std::string CLIUI::parseExpression(const std::string& expr, size_t& pos) {
    skipSpaces(expr, pos);
    
    std::string left = parseTerm(expr, pos);
    
    while (true) {
        skipSpaces(expr, pos);
        
        if (pos >= expr.length()) break;
        
        char op = expr[pos];
        if (op != '+' && op != '-') break;
        
        ++pos;
        std::string right = parseTerm(expr, pos);
        
        if (op == '+') {
            OperationResult res = ops->add(left, right);
            left = resultToString(res);
        } else {
            OperationResult res = ops->subtract(left, right);
            left = resultToString(res);
        }
    }
    
    return left;
}
\end{lstlisting}

\subsubsection*{Режим калькулятора}

Метод \texttt{calculatorMode} реализует интерактивный режим вычисления выражений.

\begin{itemize}
    \item \textbf{Вход:} арифметические выражения от пользователя (например, \texttt{g*(b+d)}).
    \item \textbf{Выход:} результаты вычислений или сообщения об ошибках.
\end{itemize}

\begin{lstlisting}[language=C++, caption=Режим калькулятора, basicstyle=\footnotesize\ttfamily]
void CLIUI::calculatorMode() {
    clearScreen();
    printHeader("РЕЖИМ КАЛЬКУЛЯТОРА");

    std::cout << "  Поддерживаются выражения со скобками!\n";
    std::cout << "  Примеры: g*(b+d), (bg+c)*g, ((a+b)+c)\n";
    std::cout << "  Операции: +, -, *, /\n";
    std::cout << "  Для выхода введите 'q'\n\n";

    printSeparator();

    std::cin.ignore(std::numeric_limits<std::streamsize>::max(), '\n');

    while (true) {
        std::cout << "\n> ";

        std::string input;
        std::getline(std::cin, input);

        if (input == "q" || input == "Q") {
            break;
        }

        try {
            size_t pos = 0;
            std::string result = parseExpression(input, pos);
            
            skipSpaces(input, pos);
            if (pos < input.length()) {
                throw std::runtime_error(
                    "Неожиданный символ: '" 
                    + std::string(1, input[pos]) + "'");
            }
            
            std::cout << "  = " << result << "\n";
            
        } catch (const std::exception& e) {
            std::cout << "  Ошибка: " << e.what() << "\n";
        }
    }
}
\end{lstlisting}

\subsubsection*{Обработка ошибок}

Метод \texttt{resultToString} преобразует результат операции в строку и обрабатывает переполнение.

\begin{itemize}
    \item \textbf{Вход:} структура \texttt{OperationResult} с результатом операции.
    \item \textbf{Выход:} строковое представление числа или исключение при переполнении.
\end{itemize}

\begin{lstlisting}[language=C++, caption=Обработка результата, basicstyle=\footnotesize\ttfamily]
std::string CLIUI::resultToString(const OperationResult& res) {
    if (res.isOverflow) {
        throw std::runtime_error("ПЕРЕПОЛНЕНИЕ");
    }
    return (res.isNegative ? "-" : "") + res.value;
}
\end{lstlisting}

При переполнении, делении на ноль или неопределённости $0/0$ выбрасывается исключение, которое перехватывается в главном цикле калькулятора и выводится пользователю.

\subsubsection*{Вспомогательные методы интерфейса}

Класс \texttt{CLIUI} содержит вспомогательные методы для работы с консолью:

\subsubsection*{Вспомогательные методы интерфейса}

Класс \texttt{CLIUI} содержит вспомогательные методы для работы с консолью:

\begin{itemize}
    \item \texttt{clearScreen()}~--- очищает экран терминала системным вызовом.
    \begin{itemize}
        \item \textbf{Вход:} состояние терминала.
        \item \textbf{Выход:} очищенный экран терминала.
    \end{itemize}
    
    \item \texttt{pause()}~--- ожидает нажатия Enter для продолжения работы.
    \begin{itemize}
        \item \textbf{Вход:} состояние ожидания ввода.
        \item \textbf{Выход:} продолжение выполнения после нажатия Enter.
    \end{itemize}
    
    \item \texttt{printHeader(title)}~--- выводит заголовок в рамке.
    \begin{itemize}
        \item \textbf{Вход:} строка заголовка.
        \item \textbf{Выход:} отформатированный заголовок в консоли.
    \end{itemize}
    
    \item \texttt{printSeparator()}~--- выводит горизонтальную линию-разделитель.
    \begin{itemize}
        \item \textbf{Вход:} состояние консоли.
        \item \textbf{Выход:} горизонтальная линия в консоли.
    \end{itemize}
    
    \item \texttt{hasSystem()}~--- проверяет, инициализирована ли система $Z_8$.
    \begin{itemize}
        \item \textbf{Вход:} указатель на объект операций.
        \item \textbf{Выход:} истина, если система инициализирована, иначе ложь.
    \end{itemize}
    
    \item \texttt{showSystemInfo()}~--- вызывает метод \texttt{printInfo()} класса $Z_8$ для отображения информации о системе.
    \begin{itemize}
        \item \textbf{Вход:} инициализированная система $Z_8$.
        \item \textbf{Выход:} информация о системе в консоли.
    \end{itemize}
    
    \item \texttt{skipSpaces(expr, pos)}~--- пропускает пробелы в строке выражения.
    \begin{itemize}
        \item \textbf{Вход:} строка выражения и текущая позиция.
        \item \textbf{Выход:} обновлённая позиция, указывающая на первый непробельный символ.
    \end{itemize}
\end{itemize}