\section*{Заключение}
\addcontentsline{toc}{section}{Заключение}

В ходе выполнения лабораторной работы была разработана программа для анализа булевых функций, а также выполнена расчётная честь задания: была построенна таблица истинности и семантическое дерево, а также бинарная диаграмма решений; найдены первая приизводная по каждой переменной и четвёртая производная; построены синтаксическое дерево и логическая схема. Реализованы все требуемые функции: построение таблицы истинности, СДНФ и СКНФ, вычисление полинома Жегалкина методом треугольника, хранение и вычисление по бинарной диаграмме решений.

\subsection*{Достоинства реализации}

\begin{itemize}
    \item Объектно-ориентированная архитектура с чётким разделением ответственности между классами \texttt{TruthTable}, \texttt{ZhegalkinPolynomial} и \texttt{BDD}.
    \item Вывод пути по БДР при вычислении значения функции, что позволяет проверить корректность работы.
    \item Модульная структура проекта с разделением на заголовочные файлы.
\end{itemize}

\subsection*{Недостатки реализации}

\begin{itemize}
    \item Структура БДР задана статически для конкретной функции, а не строится динамически.
    \item Отсутствует возможность ввода произвольной функции~--- вектор значений задан в коде.
    \item Не реализовано сохранение результатов между сеансами работы программы.
\end{itemize}

\subsection*{Масштабируемость}

Архитектура программы обеспечивает гибкость для дальнейшего развития.

\begin{itemize}
    \item Добавление новых действий над булевыми функциями сводится к реализации методов в соответствующих классах и обновлению меню. Базовая логика и структура данных остаются неизменными.
    
    Например, можно добавить методы для минимизации ДНФ, построения логических схем или вычисления производных функции.

    \item Логика работы с булевыми функциями полностью отделена от пользовательского интерфейса. Это позволяет заменить консольный интерфейс на графический без изменения классов \texttt{TruthTable}, \newline\texttt{ZhegalkinPolynomial} и \texttt{BDD}.
\end{itemize}

Таким образом, разработанная программа имеет возможности для масштабирования, что делает её пригодной для дальнейшего развития и использования в более сложных задачах, связанных с анализом булевых функций.