\section{Результаты работы программы}

\subsection{Общее описание}

В результате выполнения лабораторной работы была разработана консольная программа для анализа булевых функций. Программа полностью реализует все требования задания:

\begin{itemize}
    \item Хранение булевой функции в виде таблицы истинности.
    \item Построение СДНФ (совершенной дизъюнктивной нормальной формы).
    \item Построение СКНФ (совершенной конъюнктивной нормальной формы).
    \item Построение полинома Жегалкина методом треугольника Паскаля.
    \item Хранение и вычисление по бинарной диаграмме решений (БДР).
    \item Вычисление значения функции по полиному Жегалкина.
    \item Вычисление значения функции по БДР с выводом пути.
    \item Защита от некорректного пользовательского ввода.
\end{itemize}

Интерфейс программы построен на системе меню, позволяющей пользователю выбирать различные представления функции и выполнять вычисления.

\subsection{Демонстрация основных функций}

При запуске программы отображается главное меню (рис.~\ref{fig:Screenshot_2026-01-15_at_12.37.55.png}).

\screenshot{Screenshot_2026-01-15_at_12.37.55.png}{Главное меню программы}

При выборе пункта «Таблица истинности» выводится полная таблица значений функции для всех 16 наборов переменных (рис.~\ref{fig:Screenshot_2026-01-15_at_12.38.08.png}).

\screenshot{Screenshot_2026-01-15_at_12.38.08.png}{Таблица истинности функции}

При выборе пункта «СДНФ» программа строит и выводит совершенную дизъюнктивную нормальную форму функции (рис.~\ref{fig:Screenshot_2026-01-15_at_12.38.23.png}).

\screenshot{Screenshot_2026-01-15_at_12.38.23.png}{СДНФ функции}

При выборе пункта «СКНФ» программа строит и выводит совершенную конъюнктивную нормальную форму функции (рис.~\ref{fig:Screenshot_2026-01-15_at_12.39.08.png}).

\screenshot{Screenshot_2026-01-15_at_12.39.08.png}{СКНФ функции}

При выборе пункта «Полином Жегалкина» программа вычисляет коэффициенты методом треугольника Паскаля и выводит полином (рис.~\ref{fig:Screenshot_2026-01-15_at_12.39.17.png}).

\screenshot{Screenshot_2026-01-15_at_12.39.17.png}{Полином Жегалкина}

При выборе пункта «БДР» программа выводит структуру бинарной диаграммы решений: список узлов с их связями, порядок переменных, корень и общее количество узлов (рис.~\ref{fig:Screenshot_2026-01-15_at_12.39.50.png}).

\screenshot{Screenshot_2026-01-15_at_12.39.50.png}{Структура БДР}

\subsection{Вычисление значения функции}

При выборе пункта «Вычислить по полиному Жегалкина» программа запрашивает значения переменных и вычисляет значение функции. На рисунке~\ref{fig:Screenshot_2026-01-15_at_12.40.19.png} также показана обработка некорректного ввода: при вводе значения 3 для переменной $x_1$ и символа <<a>> для переменной $x_4$ программа выводит сообщение об ошибке и запрашивает повторный ввод.

\screenshot{Screenshot_2026-01-15_at_12.40.19.png}{Вычисление по полиному Жегалкина с валидацией ввода}

При выборе пункта «Вычислить по БДР» программа запрашивает значения переменных, выводит путь по диаграмме и результат вычисления (рис.~\ref{fig:Screenshot_2026-01-15_at_12.40.45.png}). Для набора $(1, 1, 1, 1)$ путь проходит через узлы: $x_1 = 1 \rightarrow [4] \rightarrow x_2 = 1 \rightarrow [2] \rightarrow x_3 = 1 \rightarrow [1]$, результат $f(1, 1, 1, 1) = 1$.

\screenshot{Screenshot_2026-01-15_at_12.40.45.png}{Вычисление по БДР с выводом пути}