\section{Математическое описание}

\textbf{Мультимножество.} Мультимножество~--- это обобщение понятия множества, в котором элементы могут повторяться. Каждому элементу $x_i$ сопоставлена кратность $k_i \in \mathbb{N} \cup \{0\}$, указывающая, сколько раз элемент входит в мультимножество:
\[
A = \{x_1^{k_1}, x_2^{k_2}, \ldots, x_n^{k_n}\}.
\]

\textbf{Универсум мультимножеств.} Универсум~--- это мультимножество, содержащее все возможные элементы с максимальными кратностями. Пусть задана разрядность $n$ и максимальная кратность $m$. Тогда универсум $U$:
\[
U = \{x_1^{m}, x_2^{m}, \ldots, x_n^{m}\};\; n \in \mathbb{N};\; m \in \mathbb{N} \cup \{0\},
\]
где $x_i$~--- $n$-битовый код Грея.

\textbf{Код Грея.} Код Грея~--- это система кодирования, в которой два соседних значения отличаются ровно в одном бите. Алгоритм преобразования:
\begin{itemize}
    \item Вход: натуральное число $i$.
    \item Выход: код Грея $G(i) = i \oplus \lfloor i/2 \rfloor$,
\end{itemize}
где $\oplus$~--- операция побитового исключающего ИЛИ.

\textbf{Подмножества универсума.} Любое мультимножество $A$~--- это подмножество универсума, в котором кратности элементов не превышают максимальную:
\[
A = \{x_1^{k_1}, x_2^{k_2}, \ldots, x_n^{k_n}\};\; k_i \leq m.
\]

\textbf{Операции над мультимножествами.} Для любых двух мультимножеств $A$ и $B$ с универсумом $U$ операции задаются следующим образом:

\begin{itemize}
    \item Объединение:
    \[
    A \cup B = \{x^{k_{AB}} \mid k_{AB} = \max(k_A, k_B)\},
    \]
    где $k_A$~--- кратность $x$ в $A$, $k_B$~--- в $B$.
    
    \item Пересечение:
    \[
    A \cap B = \{x^{k_{AB}} \mid k_{AB} = \min(k_A, k_B)\}.
    \]
    
    \item Разность:
    \[
     A \setminus B = A \cap \overline{B} = \{x^{k_{AB}} \mid k_{AB} = \min(k_A, m - k_B)\},
    \]
    где $m$~--- максимальная кратность элемента универсума.

    \item Симметрическая разность:
    \begin{multline*}
    A \triangle B = (A \cup B) \setminus (A \cap B) = (A \cup B) \cap \overline{(A \cap B)} = (A \cup B) \cap (\overline{A} \cup \overline{B}) = \\
    = (A \cap \overline{A}) \cup (B \cap \overline{A}) \cup (A \cap \overline{B}) \cup (B \cap \overline{B}) = \\
    = (B \cap \overline{A}) \cup (A \cap \overline{B}) = (B \setminus A) \cup (A \setminus B) = (A \setminus B) \cup (B \setminus A).
    \end{multline*}
    
    \item Дополнение:
    \[
    \overline{A} = \{x^{k_{AB}} \mid k_{AB} = m - k_A\}.
    \]
\end{itemize}

\textbf{Арифметические операции.} Арифметические операции выполняются поэлементно над кратностями:

\begin{itemize}
    \item Арифметическая сумма:
    \[
    A + B = \{x^{k_{AB}} \mid k_{AB} = \min(k_A + k_B, m)\},
    \]
    где $m$~--- максимальная кратность элемента, $k_A$~--- кратность элемента в мультимножестве $A$, $k_B$~--- кратность элемента в мультимножестве $B$.
    
    \item Арифметическая разность:
    \[
    A - B = \{x^{k_{AB}} \mid k_{AB} = \max(k_A - k_B, 0)\}.
    \]
    
    \item Арифметическое произведение:
    \[
    A \cdot B = \{x^{k_{AB}} \mid k_{AB} = \min(k_A \cdot k_B, m)\}.
    \]
    
    \item Арифметическое деление:
    \[
    A \div B = \{x^{k_{AB}} \mid k_{AB} = \lfloor k_A / k_B \rfloor\};\; k_B \neq 0.
    \]
    При $k_B = 0$ результат деления равен $0$.
\end{itemize}