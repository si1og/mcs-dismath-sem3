\section{Результаты работы программы}

\subsection{Общее описание}

В результате выполнения лабораторной работы была разработана консольная программа для работы с мультимножествами на основе кода Грея. Программа полностью реализует все требования задания:

\begin{itemize}
    \item Генерация кода Грея произвольной разрядности от 0 до 30.
    \item Создание универсума с настраиваемой максимальной кратностью элементов.
    \item Ручное и автоматическое (случайное) заполнение мультимножеств.
    \item Операции над мультимножествами: объединение, пересечение, разность, симметрическая разность, дополнение.
    \item Арифметические операции над кратностями: сумма, разность, произведение, деление.
    \item Сравнение мультимножеств на равенство.
    \item Измерение времени выполнения операций.
    \item Защита от некорректного пользовательского ввода.
    \item Адаптивный вывод: компактный для небольших множеств, постраничный для больших.
\end{itemize}

Интерфейс программы построен на системе меню, позволяющей пользователю последовательно создавать универсум, формировать мультимножества и выполнять над ними различные операции. Результаты операций могут быть сохранены как новые мультимножества для дальнейшего использования.

\subsection{Демонстрация основных функций}

При запуске программы отображается главное меню (рис.~\ref{fig:Screenshot_2025-12-03_at_09.07.01.png}). Если универсум ещё не создан, выводится соответствующее предупреждение.

\screenshot{Screenshot_2025-12-03_at_09.07.01.png}{Главное меню программы}

При выборе пункта «Создать универсум» программа запрашивает разрядность кода Грея и максимальную кратность элементов (рис.~\ref{fig:Screenshot_2025-12-03_at_09.07.14.png}).

\screenshot{Screenshot_2025-12-03_at_09.07.14.png}{Запрос параметров универсума}

После ввода параметров создаётся универсум и выводится таблица его элементов. При попытке создать новый универсум программа предупреждает о существующем и запрашивает подтверждение (рис.~\ref{fig:Screenshot_2025-12-03_at_09.08.19.png}).

\screenshot{Screenshot_2025-12-03_at_09.08.19.png}{Создание универсума с разрядностью 2 и кратностью 4}

Мультимножества можно заполнять автоматически (рис.~\ref{fig:Screenshot_2025-12-03_at_09.08.55.png}) или вручную (рис.~\ref{fig:Screenshot_2025-12-03_at_09.09.44.png}).

\screenshot{Screenshot_2025-12-03_at_09.08.55.png}{Автоматическое заполнение мультимножества}

\screenshot[0.75]{Screenshot_2025-12-03_at_09.09.44.png}{Ручное заполнение мультимножества с валидацией ввода}

После создания мультимножеств доступны операции над ними. На рисунке~\ref{fig:Screenshot_2025-12-03_at_09.11.26.png} показан результат операции разности.

\screenshot[0.75]{Screenshot_2025-12-03_at_09.11.26.png}{Выполнение операции разности мультимножеств}

\subsection{Обработка пользовательского ввода}

Программа обеспечивает защиту от некорректного ввода на всех этапах работы. При ручном заполнении мультимножества проверяется принадлежность элемента универсуму и отсутствие дубликатов (рис.~\ref{fig:Screenshot_2025-12-03_at_09.09.44.png}). На рисунке видно, что при попытке добавить уже существующий элемент «00» программа выводит сообщение об ошибке и запрашивает повторный ввод.

При выборе мультимножеств для операций программа проверяет их существование (рис.~\ref{fig:Screenshot_2025-12-03_at_09.10.46.png}).

\screenshot{Screenshot_2025-12-03_at_09.10.46.png}{Обработка ошибки: мультимножество не найдено}

\subsection{Работа программы при большой размерности универсума}

Программа корректно работает с универсумами большой размерности. При превышении рекомендуемого значения (20 бит) выводится предупреждение о возможном большом потреблении памяти и времени выполнения.

На рисунке~\ref{fig:Screenshot_2025-12-03_at_09.12.46.png} показано создание универсума с разрядностью 21, содержащего более 2 миллионов элементов. Для больших универсумов используется постраничный вывод: отображаются первые и последние 10 элементов с указанием количества пропущенных.

\screenshot[0.75]{Screenshot_2025-12-03_at_09.12.46.png}{Создание универсума с разрядностью 21 (2\,097\,152 элемента)}

При автоматическом заполнении большого мультимножества программа измеряет время выполнения и выводит предупреждение, если операция заняла более 1 секунды (рис.~\ref{fig:Screenshot_2025-12-03_at_09.13.15.png}).

\screenshot{Screenshot_2025-12-03_at_09.13.15.png}{Создание большого мультимножества с предупреждением о времени выполнения}

\subsection{Работа с пустым универсумом}

Программа поддерживает создание пустого универсума при нулевой разрядности (рис.~\ref{fig:Screenshot_2025-12-03_at_09.15.47.png}). В этом случае максимальная кратность автоматически устанавливается в 0.

\screenshot{Screenshot_2025-12-03_at_09.15.47.png}{Создание пустого универсума}

При создании мультимножества в пустом универсуме автоматически создаётся пустое мультимножество (рис.~\ref{fig:Screenshot_2025-12-03_at_09.15.59.png}, \ref{fig:Screenshot_2025-12-03_at_09.16.10.png}).

\screenshot{Screenshot_2025-12-03_at_09.15.59.png}{Создание пустого мультимножества «a»}

\screenshot{Screenshot_2025-12-03_at_09.16.10.png}{Создание пустого мультимножества «b»}

Все операции над пустыми мультимножествами корректно возвращают пустые результаты (рис.~\ref{fig:Screenshot_2025-12-03_at_09.16.44.png}).

\screenshot[0.65]{Screenshot_2025-12-03_at_09.16.44.png}{Результаты всех операций над пустыми мультимножествами}