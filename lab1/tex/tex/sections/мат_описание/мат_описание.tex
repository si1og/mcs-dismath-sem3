\section{Математическое описание}

\textbf{Код Грея.} Код Грея~--- это система кодирования, в которой два соседних значения отличаются ровно в одном бите. Для преобразования натурального числа $i$ в код Грея $G(i)$ используется формула:
\[
G(i) = i \oplus \lfloor i/2 \rfloor,
\]
где $\oplus$~--- операция побитового исключающего ИЛИ (XOR).

В данной работе код Грея используется для генерации элементов универсума: каждому числу от $0$ до $2^n - 1$ ставится в соответствие $n$-битовая строка.

\textbf{Универсум мультимножеств.} Пусть задана разрядность $n$ и максимальная кратность $m$. Универсум $U$:
\[
U = \{x_1^{m_1}, x_2^{m_2}, \ldots, x_n^{m_n}\};\; n \in \mathbb{N};\; m \in \mathbb{N} \cup \{0\},
\]
где $x_i$~--- $n$-битовый код Грея с кратностью $m_i$.

\textbf{Мультимножества.} Мультимножество~--- подмножество универсума, которое задаётся следующим способом:
\[
A = \{x_1^{k_1}, x_2^{k_2}, \ldots, x_n^{k_n}\};\; k_i \leq m_i.
\]

\textbf{Операции над мультимножествами.} Для любых двух мультимножеств $A$ и $B$ с универсумом $U$ операции задаются следующим образом:

\begin{itemize}
    \item Объединение:
    \[
    A \cup B = \{x^{k_{AB}} \mid k_{AB} = \max(k_A, k_B)\},
    \]
    где $k_A$~--- кратность $x$ в $A$, $k_B$~--- в $B$.
    
    \item Пересечение:
    \[
    A \cap B = \{x^{k_{AB}} \mid k_{AB} = \min(k_A, k_B)\}.
    \]
    
    \item Разность:
    \[
    A \setminus B = A \cap \overline{B}.
    \]
    
    \item Симметрическая разность:
    \begin{multline*}
    A \triangle B = (A \cup B) \setminus (A \cap B) = (A \cup B) \cap \overline{(A \cap B)} = (A \cup B) \cap (\overline{A} \cup \overline{B}) = \\
    = (A \cap \overline{A}) \cup (B \cap \overline{A}) \cup (A \cap \overline{B}) \cup (B \cap \overline{B}) = \\
    = (B \cap \overline{A}) \cup (A \cap \overline{B}) = (B \setminus A) \cup (A \setminus B) = (A \setminus B) \cup (B \setminus A).
    \end{multline*}
    
    \item Дополнение:
    \[
    \overline{A} = \{x^{k_{AB}} \mid k_{AB} = m - k_A\},
    \]
    где $m$~--- максимальная кратность элемента универсума.
\end{itemize}

\textbf{Арифметические операции.} Арифметические операции выполняются поэлементно над кратностями:

\begin{itemize}
    \item Арифметическая сумма:
    \[
    A + B = \{x_{AB}^{k} \mid k_{AB} = \min(k_A + k_B, m)\},
    \]
    где $m$~--- максимальная кратность элемента, $k_A$~--- кратность элемента в мультимножестве $A$, $k_B$~--- кратность элемента в мультимножестве $B$.
    
    \item Арифметическая разность:
    \[
    A - B = \{x^{k_{AB}} \mid k_{AB} = \max(k_A - k_B, 0)\}.
    \]
    
    \item Арифметическое произведение:
    \[
    A \cdot B = \{x^{k_{AB}} \mid k_{AB} = \min(k_A \cdot k_B, m)\}.
    \]
    
    \item Арифметическое деление:
    \[
    A \div B = \left\{x^{k_{AB}} \mid k_{AB} = \max\left(\left\lfloor \frac{k_A}{k_B} \right\rfloor, 0\right)\right\},
    \]
    при $k_B = 0$ результат равен $0$.
\end{itemize}

Таким образом, программа должна генерировать универсум на основе кода Грея заданной разрядности, формировать мультимножества с указанием кратностей элементов и выполнять над ними все перечисленные теоретико-множественные и арифметические операции.