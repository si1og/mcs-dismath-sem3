\section*{Заключение}
\addcontentsline{toc}{section}{Заключение}

В ходе выполнения лабораторной работы была разработана программа для работы с мультимножествами на основе кода Грея. Реализованы все требуемые функции: генерация кода Грея произвольной разрядности, создание универсума и мультимножеств, операции над мульминожествами (объединение, пересечение, разность, симметрическая разность, дополнение), арифметические операции над кратностями (сумма, разность, произведение, деление), а также сравнение мультимножеств на равенство.

\subsection{Достоинства реализации}

\begin{itemize}
    \item Объектно-ориентированная архитектура с чётким разделением ответственности между классами \texttt{Universe}, \texttt{Multiset} и \texttt{CLIUI}.
    \item Использование современных возможностей C++17: умные указатели (\texttt{std::unique\_ptr}), библиотека \texttt{<chrono>} для измерения времени.
    \item Защита от некорректного пользовательского ввода на всех этапах работы программы.
    \item Адаптивный вывод: компактные таблицы для небольших множеств, постраничный режим для больших.
    \item Возможность сохранения результатов операций как новых мультимножеств для дальнейшего использования.
    \item Измерение и отображение времени выполнения операций с предупреждениями о долгих вычислениях.
\end{itemize}

\subsection{Недостатки реализации}

\begin{itemize}
    \item Линейный поиск в методе \texttt{contains} класса \texttt{Universe} имеет сложность $O(n)$, что может быть оптимизировано использованием хеш-таблицы.
    \item Созданные мультимножества не сохраняются между сеансами работы программы.
    \item Консольный интерфейс ограничивает удобство работы с большими множествами.
    \item Некоторые операции создают копии мультимножеств, что увеличивает потребление памяти.
\end{itemize}

\subsection{Масштабируемость}

Программа демонстрирует следующие характеристики масштабируемости:

\begin{itemize}
    \item При размерности до 20 бит ($2^{20} \approx 1$~млн элементов) программа работает быстро, и изредка время выполнения операций может превышать 1--2 секудны.
    \item При размерности 21 бит ($2^{21} \approx 2$~млн элементов) время создания и случайного заполнения мультимножеств составляет 3--4 секунды.
    \item Арифметические операции над множествами с 600\,000--1\,600\,000 элементов выполняются за 4--5 секунд.
    \item Жёсткое ограничение размерности составляет 30 бит, что обусловлено разрядностью типа \texttt{int}.
\end{itemize}

Таким образом, программа пригодна для практического использования при работе с мультимножествами умеренного размера и может служить основой для дальнейшего развития с учётом выявленных недостатков.