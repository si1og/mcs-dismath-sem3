\section*{Введение}
\addcontentsline{toc}{section}{Введение}

Код Грея~--- способ кодирования последовательных чисел, при котором соседние значения отличаются ровно в одном бите. Это свойство минимизирует ошибки при переходе между состояниями и применяется в цифровой электронике, системах передачи данных и адресации.

Мультимножества обобщают классические множества, допуская повторение элементов с заданной кратностью. Они используются в базах данных, статистике и комбинаторике.

В данной работе реализована программа генерации бинарного кода Грея для заполнения универсума мультимножеств, а также выполнения теоретико-множественных и арифметических операций над ними.
